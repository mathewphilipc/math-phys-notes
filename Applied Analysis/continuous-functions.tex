\documentclass[12 pt]{article}
\usepackage{amsmath, amssymb, mathtools, slashed}

% Commonly used sets of numbers
\newcommand{\R}{\mathbb{R}}
\newcommand{\Z}{\mathbb{Z}}
\newcommand{\C}{\mathbb{C}}
\newcommand{\Q}{\mathbb{Q}}
\newcommand{\N}{\mathbb{N}}

% Shortcuts for inner product spaces
\newcommand{\KET}[1]{\left| #1 \right\rangle }
\newcommand{\BRA}[1]{\left\langle #1 \right| }
\newcommand{\IP}[2]{\left\langle #1 \left| #2 \right\rangle \right.}
\newcommand{\Ip}[2]{\left\langle #1, #2 \right\rangle}
\newcommand{\nm}[1]{\left\| #1 \right\|}

% Shortcuts for the section of 3D rotations
\newcommand{\lo}{\textbf{L}_1}
\newcommand{\ltw}{\textbf{L}_2}
\newcommand{\lt}{\textbf{L}_3}

% Shortcuts for geometric vectors
\newcommand{\U}{\textbf{u}}
\newcommand{\V}{\textbf{v}}
\newcommand{\W}{\textbf{w}}
\newcommand{\B}[1]{\mathbf{#1}}
\newcommand{\BA}[1]{\hat{\mathbf{#1}}}

% Other shortcuts

\newcommand{\G}{\gamma}
\newcommand{\LA}{\mathcal{L}}
\newcommand{\X}{\Vec{x}}
\newcommand{\x}{\Vec{x}}

\newcommand{\LP}{\left(}
\newcommand{\RP}{\right)}

\newcommand{\DI}{\mbox{dist}}

\newcommand{\PA}[2]{\frac{\partial #1}{\partial #2}}

\newcommand{\HI}{\mathcal{H}}
\newcommand{\AL}{\mathcal{A}}

\newcommand{\D}{\partial}

\newcommand{\bs}{\textbackslash}

\newcommand{\T}{\mathcal{T}}

\numberwithin{equation}{section}
\setcounter{section}{0}





\def\Xint#1{\mathchoice
{\XXint\displaystyle\textstyle{#1}}%
{\XXint\textstyle\scriptstyle{#1}}%
{\XXint\scriptstyle\scriptscriptstyle{#1}}%
{\XXint\scriptscriptstyle\scriptscriptstyle{#1}}%
\!\int}
\def\XXint#1#2#3{{\setbox0=\hbox{$#1{#2#3}{\int}$ }
\vcenter{\hbox{$#2#3$ }}\kern-.6\wd0}}
\def\ddashint{\Xint=}
\def\dashint{\Xint-}



\begin{document}


\title{Continuous Functions}
\author{Mathew Calkins\\
  \texttt{mathewpcalkins@gmail.com}}

\date{\today}

\maketitle

\tableofcontents


\section{Convergence of functions}

Given a metric space $(X, d)$ and a collection of functions $f_n: X \to \R$, we distinguish between two notions of convergence to $f: X \to \R$. \begin{itemize}
\item Pointwise convergence: for all $x \in X$, $\lim_{n \to \infty} f_n(x) = f(x)$.
\item Metric convergence: $\lim_{n \to \infty} \nm{f_n - f} = 0$ for some metric $\nm{\cdot}$ on the space of functions from which we draw each $f_n$.
\end{itemize}
By considering the sequence of maps $f_n: [0,1] \to \R$ given by $x \mapsto x^n$, we see that a pointwise-convergent sequence of continuous functions may have a discontinuous limit. So this is a bad notion of convergence if we want to restrict ourselves to continuous functions.\\
\\
\
For continuous functions $f: X \to \R$, a natural norm on spaces of continuous functions is the \textit{uniform} or \text{sup} norm \begin{equation*}
\nm{f} = \sup _{x \in X} |f(x)|.
\end{equation*}
Imagining the behavior of the $L^p$ norm for large $p$, this is sometimes written $\nm{\cdot}_\infty$.\\
\\
\
\textbf{Definition 2.2} A sequence of bounded functions $f_n: X \to \R$ \textit{converges uniformly} to a function $f$ if \begin{equation*}
\lim_{n \to \infty} \nm{f_n - f} = 0.
\end{equation*}
\textbf{Theorem 2.3} Let $(f_n)$ be a sequence of bounded, continuous, real-valued functions on a metric space $(X,d)$. If $f_n \to f$ uniformly, then $f$ is continuous.\\
\\
\
\textbf{Proof} To prove this claim we don't need the full machinery of a sequence of function converging uniformly to $f$. We work with a weaker premise which better captures the essence of why our theorem is true: the difference $\nm{f - g}$ can be made arbitrarily small by appropriate choice of continuous $g: X \to \R$. Morally, our premise is that we have a function $f: X \to \R$ which can be approximated arbitrarily well (if we measure approximations using the uniform norm) by continuous functions.\\
\\
\
Morals aside, we return to our claim that $f: X \to \R$ is continous. Fixing $\epsilon > 0$, we want to find $\delta > 0$ such that \begin{equation*}
d(x,y) < \delta \implies |f(x) - f(y)| < \epsilon.
\end{equation*}
From the premise, there must exist continuous $g: X \to \R$ such that \begin{equation*}
\nm{f - g} = \sup_{x \in X} |f(x) - g(x)| < \epsilon / 3.
\end{equation*}
Pick such a function $g$. From the continuity of $g$, there exists $\delta > 0$ such that \begin{equation*}
d(x,y) < \delta \implies |g(x) - g(y)| < \epsilon / 3.
\end{equation*}
In that case, when $d(x,y) < \delta$ we have \begin{align*}
|f(x) - f(y)| & = |(f(x) - g(x)) + (g(x) - g(y)) + (g(y) - f(y))| \\
\ & \leq |f(x) - g(x)| + |g(x) - g(y)| + |g(y) - f(y)| \\ 
 & < \epsilon / 3 + \epsilon / 3 + \epsilon / 3 \\
\ & = \epsilon.
\end{align*}
QED




\section{Spaces of continuous functions}
Given a metric space $(X, d)$, we denote by $C(X)$ the space of continuous functions $f: X \to \R$. This is a real linear space under the obvious operations.\\
\\
\
\textbf{Theorem 2.4} Let $(K,d)$ be a compact metric space. Then $C(K)$ is complete.









\section{Approximation by polynomials}











\section{Compact subsets of $C(K)$}







\section{Ordinary differential equations}






\section{Exercises}

\textbf{Exercise 2.2 (November 29)} Let $f_n \in C([a,b])$ be a sequence of functions converging uniformly to a function $f$. Show that \begin{equation*}
\lim_{n \to \infty} \int_a ^b f_n(x)dx = \int_a ^b f(x)dx.
\end{equation*}
Give a counterexample to show that the pointwise convergence of continuous functions $f_n$ to a continuous function $f$ does not imply the convergence of the corresponding integrals.\\
\\
\
\textbf{Solution} First we recall the generic bound \begin{equation*}
\left| \int_a^b g(x) dx \right| \leq \int_a ^b |g(x)|dx
\end{equation*}
for $g \in C([a,b])$ (i.e., for continous $g: [a,b] \to \R$). To show that \begin{equation*}
\lim_{n \to \infty} \int_a ^b f_n(x)dx = \int_a ^b f(x)dx.
\end{equation*}
we equivalently show that \begin{equation*}
\lim_{n \to \infty} \left| \int_a ^b f_n(x)dx - \int_a ^b f(x)dx \right| = 0.
\end{equation*}
To do so, observe the sequence of upper bounds \begin{align*}
\lim_{n \to \infty} \left| \int_a ^b f_n(x)dx - \int_a ^b f(x)dx \right| & = \lim_{n \to \infty} \left| \int_a ^b \LP f_n(x) - f(x)\RP dx\right| \\
\ & \leq \lim_{n \to \infty} \int_a ^b \left| f_n(x) - f(x) \right|dx \\
\ & \leq \lim_{n \to \infty} \LP (b - a) \cdot \sup_{x \in [a,b]} |f_n(x) - f(x)| \RP \\
\ & = (b-a) \cdot  \lim_{n \to \infty} \LP \sup_{x \in [a,b]} |f_n(x) - f(x)| \RP 
\end{align*}
So it suffices to show that \begin{equation*}
 \lim_{n \to \infty} \LP \sup_{x \in [a,b]} |f_n(x) - f(x)| \RP.
\end{equation*}
But that is precisely the definition of the statement ``$f_n \to f$ uniformly" so we are done. QED\\
\\
\
(still thinking about the counterexample for the other claim in the problem statement)

\end{document}
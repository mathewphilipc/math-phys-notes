\documentclass[12 pt]{article}
\usepackage{amsmath, amssymb, mathtools, slashed}

% Commonly used sets of numbers
\newcommand{\R}{\mathbb{R}}
\newcommand{\Z}{\mathbb{Z}}
\newcommand{\C}{\mathbb{C}}
\newcommand{\Q}{\mathbb{Q}}
\newcommand{\N}{\mathbb{N}}

% Shortcuts for inner product spaces
\newcommand{\KET}[1]{\left| #1 \right\rangle }
\newcommand{\BRA}[1]{\left\langle #1 \right| }
\newcommand{\IP}[2]{\left\langle #1 \left| #2 \right\rangle \right.}
\newcommand{\Ip}[2]{\left\langle #1, #2 \right\rangle}
\newcommand{\nm}[1]{\left\| #1 \right\|}

% Shortcuts for the section of 3D rotations
\newcommand{\lo}{\textbf{L}_1}
\newcommand{\ltw}{\textbf{L}_2}
\newcommand{\lt}{\textbf{L}_3}

% Shortcuts for geometric vectors
\newcommand{\U}{\textbf{u}}
\newcommand{\V}{\textbf{v}}
\newcommand{\W}{\textbf{w}}
\newcommand{\B}[1]{\mathbf{#1}}
\newcommand{\BA}[1]{\hat{\mathbf{#1}}}

% Other shortcuts

\newcommand{\G}{\gamma}
\newcommand{\LA}{\mathcal{L}}
\newcommand{\X}{\Vec{x}}
\newcommand{\x}{\Vec{x}}

\newcommand{\LP}{\left(}
\newcommand{\RP}{\right)}

\newcommand{\DI}{\mbox{dist}}

\newcommand{\PA}[2]{\frac{\partial #1}{\partial #2}}

\newcommand{\HI}{\mathcal{H}}
\newcommand{\AL}{\mathcal{A}}

\newcommand{\D}{\partial}

\newcommand{\bs}{\textbackslash}

\newcommand{\T}{\mathcal{T}}

\numberwithin{equation}{section}
\setcounter{section}{0}





\def\Xint#1{\mathchoice
{\XXint\displaystyle\textstyle{#1}}%
{\XXint\textstyle\scriptstyle{#1}}%
{\XXint\scriptstyle\scriptscriptstyle{#1}}%
{\XXint\scriptscriptstyle\scriptscriptstyle{#1}}%
\!\int}
\def\XXint#1#2#3{{\setbox0=\hbox{$#1{#2#3}{\int}$ }
\vcenter{\hbox{$#2#3$ }}\kern-.6\wd0}}
\def\ddashint{\Xint=}
\def\dashint{\Xint-}



\begin{document}


\title{Continuous Functions}
\author{Mathew Calkins\\
  \texttt{mathewpcalkins@gmail.com}}

\date{\today}

\maketitle


\section{Convergence of functions}

Given a metric space $(X, d)$ and a collection of functions $f_n: X \to \R$, we distinguish between two notions of convergence to $f: X \to \R$. \begin{itemize}
\item Pointwise convergence: for all $x \in X$, $\lim_{n \to \infty} f_n(x) = f(x)$.
\item Metric convergence: $\lim_{n \to \infty} \nm{f_n - f} = 0$ for some metric $\nm{\cdot}$ on the space of functions from which we draw each $f_n$.
\end{itemize}
By considering the sequence of maps $f_n: [0,1] \to \R$ given by $x \mapsto x^n$, we see that a pointwise-convergent sequence of continuous functions may have a discontinuous limit. So this is a bad notion of convergence if we want to restrict ourselves to continuous functions.\\
\\
\
For continuous functions $f: X \to \R$, a natural norm on spaces of continuous functions is the \textit{uniform} or \text{sup} norm \begin{equation*}
\nm{f} = \sup _{x \in X} |f(x)|.
\end{equation*}
Imagining the behavior of the $L^p$ norm for large $p$, this is sometimes written $\nm{\cdot}_\infty$.





\section{Spaces of continuous functions}









\section{Approximation by polynomials}











\section{Compact subsets of $C(K)$}







\section{Ordinary differential equations}


\end{document}
\documentclass[12 pt]{article}
\usepackage{amsmath, amssymb, mathtools, slashed}

% Commonly used sets of numbers
\newcommand{\R}{\mathbb{R}}
\newcommand{\Z}{\mathbb{Z}}
\newcommand{\C}{\mathbb{C}}
\newcommand{\Q}{\mathbb{Q}}
\newcommand{\N}{\mathbb{N}}

% Shortcuts for inner product spaces
\newcommand{\KET}[1]{\left| #1 \right\rangle }
\newcommand{\BRA}[1]{\left\langle #1 \right| }
\newcommand{\IP}[2]{\left\langle #1 \left| #2 \right\rangle \right.}
\newcommand{\Ip}[2]{\left\langle #1, #2 \right\rangle}
\newcommand{\nm}[1]{\left\| #1 \right\|}

% Shortcuts for the section of 3D rotations
\newcommand{\lo}{\textbf{L}_1}
\newcommand{\ltw}{\textbf{L}_2}
\newcommand{\lt}{\textbf{L}_3}

% Shortcuts for geometric vectors
\newcommand{\U}{\textbf{u}}
\newcommand{\V}{\textbf{v}}
\newcommand{\W}{\textbf{w}}
\newcommand{\B}[1]{\mathbf{#1}}
\newcommand{\BA}[1]{\hat{\mathbf{#1}}}

% Other shortcuts

\newcommand{\G}{\gamma}
\newcommand{\LA}{\mathcal{L}}
\newcommand{\X}{\Vec{x}}
\newcommand{\x}{\Vec{x}}

\newcommand{\LP}{\left(}
\newcommand{\RP}{\right)}

\newcommand{\DI}{\mbox{dist}}

\newcommand{\PA}[2]{\frac{\partial #1}{\partial #2}}

\newcommand{\HI}{\mathcal{H}}
\newcommand{\AL}{\mathcal{A}}

\newcommand{\D}{\partial}

\newcommand{\bs}{\textbackslash}

\newcommand{\T}{\mathcal{T}}

\numberwithin{equation}{section}
\setcounter{section}{0}





\def\Xint#1{\mathchoice
{\XXint\displaystyle\textstyle{#1}}%
{\XXint\textstyle\scriptstyle{#1}}%
{\XXint\scriptstyle\scriptscriptstyle{#1}}%
{\XXint\scriptscriptstyle\scriptscriptstyle{#1}}%
\!\int}
\def\XXint#1#2#3{{\setbox0=\hbox{$#1{#2#3}{\int}$ }
\vcenter{\hbox{$#2#3$ }}\kern-.6\wd0}}
\def\ddashint{\Xint=}
\def\dashint{\Xint-}



\begin{document}


\title{Notes on Metric and Normed Spaces}
\author{Mathew Calkins\\
  \texttt{mathewpcalkins@gmail.com}}

\date{\today}

\maketitle

\tableofcontents{

\section{Metrics and norms}

\begin{enumerate}
\item $d(x,y) \geq 0$
\end{enumerate}

\textbf{Claim} The unit ball in any normed linear space is convex.\\
\\
\
\textbf{Proof} Let $X$ be a linear space with norm $\nm{\cdot}$. As usual, denote by $\overline{B}$ the unit ball \begin{equation*}
\overline{B} = \{ x \in X \ : \ \nm{x} \leq 1 \}.
\end{equation*}
To demonstrate convexity, fix arbitrary $x, y \in \overline{B}$ and $t \in [0, 1]$. We claim that \begin{equation*}
tx + (1-t)y \in \overline{B}
\end{equation*}
or equivalently that \begin{equation*}
\nm{tx + (1-t)y} \leq 1.
\end{equation*}
Indeed, \begin{align*}
\nm{tx + (1-t)y} & \leq \nm{tx} + \nm{(1-t)y} \\
\ & = t \nm{x} + (1-t) \nm{y} \\
\ & \leq t \cdot 1 + (1-t) \cdot 1 \\
\ & = 1.
\end{align*}
QED\\
\\
\
\textbf{Claim} If $(X, \nm{\cdot})$ is a normed linear space, then \begin{equation*}
d(x,y) = \frac{\nm{x-y}}{1 + \nm{x-y}}
\end{equation*}
defined a nonhomogenous, translation-invariant metric on $X$.\\
\\
\
\textbf{Proof} We proof that $d: X \times X \to \R$ is a metric one property at a time. Since the top and bottom of our fraction are always nonnegative, we have that $d(x,y) \geq 0$ for all $x, y \in X$. And the only time $d(x,y) = 0$ is when $\nm{x-y} = 0$, which occurs only when $x = y$, so $d(x,y) = 0 \iff x = y$. Symmetry ($d(x,y) = d(y,x)$) is clear by inspection.\\
\\
\
The triangle inequality is harder. Fix $x, y, z \in X$. We wish to show that \begin{equation*}
\frac{\nm{x-y}}{1 + \nm{x-y}} + \frac{\nm{y-z}}{1 + \nm{y-z}} \geq \frac{\nm{x-z}}{1 + \nm{x-z}}.
\end{equation*}

\section{Convergence}


\textbf{Definition 1.12} A sequence $(x_n)$ in a metric space $(X, d)$ is \textit{Cauchy} if $\forall \epsilon > 0, \exists N$ such that $m, n \geq N \in \N \implies d(x_n, x_m) < \epsilon$.\\
\\
\
\textbf{Definition 1.16} (Usual definition of convergence in a metric space)\\
\\
\
\textbf{Claim} In a general metric space, every convergent sequence in is Cauchy.\\
\\
\
\textbf{Proof} Let $(X, \nm{\cdot})$ be a metric space and let $\{x_n\}$ be a sequence in $X$ converging to $x \in X$. Fix $\epsilon > 0$. So there exists $N$ such that $n \geq N \implies d(x, x_n) < \epsilon/2$. Then for all $m, n \geq N$ the triangle inequality implies \begin{align*}
d(x_n, x_m) & \leq d(x, x_n) + d(x, x_m) \\
\ & < \epsilon/2 + \epsilon / 2 \\
\ & = \epsilon.
\end{align*}
QED\\
\\
\
\textbf{Definition 1.17} A metric space $(X, d)$ is \textit{complete} if every Cauchy sequence in $X$ converges to a limit in $X$. A subset $Y$ is \textit{complete} if the metric subspace $(Y, d|_Y)$ is complete. A \textit{Banach space} is a normed linear space which is complete with respect to the norm-induced metric.















\section{Upper and lower bounds}

\textbf{Definition 1.20} (defitions of \textit{upper bound, lower bound, bounded from above, bounded from below} for subsets of $\R$)\\
\\
\
\textbf{Definition 1.121} (definitions of \textit{supremum / least upper bound} and \textit{infimum / greatest lower bound})\\
\\
\
Note that Hunter uses \textit{monotone increasing} to mean \textit{non-decreasing} ($n > m \implies x_n \geq x_m$) and likewise for \textit{monotone decreasing}.\\
\\
\
Given a sequence $(x_n)$ in $\R$, we define \begin{equation*}
\mbox{lim sup } x_n = \lim_{n \to \infty} \left[ \sup \{ x_k \ | \ k \geq n \} \right].
\end{equation*}
Notice that the sequence $(y_n)$ on the inside of the RHS given by \begin{equation*}
y_n = \sup \{ x_k \ | \ k \geq n \} 
\end{equation*}
is monotone increasing (i.e., never decreasing). We similarly define \begin{equation*}
\mbox{lim inf } x_n = \lim_{n \to \infty} \left[ \inf \{ x_k \ | \ k \geq n \} \right].
\end{equation*}
Both of these values always exist, as long as we allow $\pm \infty$ in addition to real values. Observe the useful sandwiching \begin{equation*}
\sup \{ x_k \ | \ k \geq 1 \} \geq \sup \{ x_k \ | \ k \geq n \}  \geq x_n \geq \inf \{ x_k \ | \ k \geq n \}  \geq \inf \{ x_k \ | \ k \geq 1 \} .
\end{equation*}
Notice that $(x_n)$ is convergent if and only if lim inf $ x_n = $ lim sup $x_n$, in which case it converges to their common value.
















\section{Continuity}

\textbf{Definition 1.26 (continuity in a metric space)} $f: X \to Y$ is \textit{continuous} at $x_0 \in X$ if $\forall \epsilon >0$, there exists $\delta > 0$ such that \begin{equation*}
d_X(x, x_0) < \delta \implies d_Y(f(x), f(x_0)) < \epsilon.
\end{equation*}
$f:X \to Y$ is \textit{continuous on} $X$ if it is continuous at every points in $X$.\\
\\
\
\textbf{Example 1.27 (distance is continuous)} Fix $a \in X$ and define $f: X \to \R$ by $f(x) = d(x, a)$. Then $f$ is continuous on $A$.\\
\\
\
\textbf{Proof} Per the premise, let $(X, d)$ be a metric space with some $a \in X$ and define $f: X \to \R$ by the rule $f(x) = d(x,a)$. To show continuity, fix arbitrary $x_0 \in X$ and $\epsilon > 0$. Being carefully about which metrics are used where, we wish to find $\delta > 0$ such that \begin{equation*}
d_X(x, x_0) < \delta \implies d_{\R} (f(x), f(x_0)) < \epsilon.
\end{equation*}
From the definition of $f$, this is equivalent to the condition \begin{equation*}
d_X(x, x_0) < \delta \implies |d_X(x, a) - d_X(x_0,a)| < \epsilon.
\end{equation*}
But according to the following corrolary of the triangle inequality: \begin{equation*}
d(x_1, x_2) \geq |d(x_1, x_3) - d(x_2, x_3)|.
\end{equation*}
It suffices to set $\delta = \epsilon$. QED\\
\\
\
\textbf{Definition 1.30 (uniform continuity in a metric space)} A function $f: X \to Y$ is \textit{uniformly continuous} on $X$ if for all $\epsilon > 0$, there exists $\delta > 0$ such that 
\begin{equation*}
d_X(x,y) < \delta \implies d_Y(f(x), f(y)) < \epsilon
\end{equation*}
for all $x, y \in X$. This differs from regular continuity in that $\delta$ is independent of  $x, y \in X$.\\
\\
\
\textbf{Example 1.32} A function $f: \R^n \to \R^m$ is \textit{affine} if \begin{equation*}
f(tx + (1-t)y) = t f(x) + (1-t)f(y) \ \forall x, y \in \R^n \mbox{ and } t \in [0,1].
\end{equation*}
Every affine function is uniformly continuous and has the form $x \mapsto Ax + b$ for some constant matrix $M$ and constant vector $b$.\\
\\
\
\textbf{Definition 1.33} A function $f: X \to Y$ is \textit{sequentially continuous} at $x \in X$ if, for every sequence $(x_n)$ converging to $x \in X$, the sequence $(f(x_n))$ converges to $f(x) \in Y$.\\
\\
\
\textbf{Proposition 1.34} Let $X, Y$ be metric spaces. Then $f: X \to Y$ is continuous at $x \in X$ if and only if it is sequentially continuous at that point.











\section{Open and closed sets}


Open and closed balls in a metric space $(X,d)$: \begin{align*}
B_r(a) & = \{ x \in X \ | \ d(x,a) < r \}, \\
\overline{B}_r(a) & = \{ x \in X \ | \ d(x,a) \geq r \}.
\end{align*}
\textbf{Definition 1.36 (open and closed sets in a metric space)} Let $(X, d)$ be a metric space. A subset $G \subset X$ is \textit{open} if $\forall x \in G$, there exists $r > 0$ such that $B_r(x) \subset G$. A subset $F \subset X$ is \textit{closed} if its complement $F^c = X - F$ is open.\\
\\
\
\textbf{Example 1.39 (rationals have Lebesgue measure zero)} Let $\{q_n \ | \ n \in \N\}$ be an enumeration of $\Q$ and fix $\epsilon > 0$. For each $n \in N$, consider the open interval \begin{equation*}
I_n = \LP q_n - \frac{\epsilon}{2^n}, q_n + \frac{\epsilon}{2^n} \RP.
\end{equation*}
The union $\cup_{n \in \N} I_n$ is very interesting: it covers $\Q$, and yet the sum of the lengths of the intervals is $2\epsilon$, where $\epsilon$ can be made as small as we want. A subset of $\R$ has \textit{Lebesgue measure zero} if $\forall \epsilon > 0$, there exists a countable collection of open intervals whose unions contains the subset and such that the sum of the lengths of the intervals is less than $\epsilon$.\\
\\
\
\textbf{Proposition 1.41 (closed sets in terms of sequences)} A subset $F$ of a metric space is closed if and only if every convergent sequence of elements of $F$ has its limit in $F$. That is, if $x_n \to x$ and $x_n \in F \ \forall n \in \N$, then $x \in F$.\\
\\
\
\textbf{Definition (closure)} In general topology, the closure $\overline{A}$ of a subset $A \subset X$ is the intersection of all closed sets containing $A$ (a sort of minimal closed superset of $A$). In a metric space, we can construct this set by appending to $A$ every point in $X$ that can be reached as a limit of points in $A$. \begin{equation*}
\overline{A} = \{ x \in X \ | \mbox{ there exists a sequence } (a_n) \mbox{ in } X \mbox{ with } a_n \to x \}.
\end{equation*}
\textbf{Definition 1.43 (dense subset)} A subset $A$ of a metric space $X$ is \textit{dense} in $X$ if $\overline{A} = X$.\\
\\
\
\textbf{Definition 1.44} A metric space is \textit{separable} if it contains a countable dense subset.\\
\\
\
(more definitions to come)












\section{The completion of a metric space}
In this section we discuss how to produce a complete metric space from an incomplete one.\\
\\
\
\textbf{Definition 1.49} A map $\iota: X \to Y$ is called an \textit{isometry} or an \textit{isometric embedding} of $X$ into $Y$ if it satisfies \begin{equation*}
d_Y(\iota(x_1), \iota(x_2)) = d_X(x_1, x_2)
\end{equation*}
for all $x_1, x_2 \in X$. If $\iota: X \to Y$ is onto (and thus bijective) we call it a \textit{metric space isomorphism} or more briefly an \textit{isomorphism} if the metric structure is clear from context. As usual, if such an $\iota$ exists we say that $X$ and $Y$ are \textit{isomorphic}.\\
\\
\
Note that an isometry is automatically continuous (simply assign $\delta = \epsilon$ in the definition of continuity).\\
\\
\
\textbf{Example} The map $\iota: \C \to \R^2$ with $x+iy \mapsto (x,y)$ is a metric spac isomorphism.\\
\\
\
\textbf{Definition 1.51} A metric space $(\tilde{X}, \tilde{d})$ is called the \textit{completion} of the metric space $(X, d)$ if the following conditions hold: \begin{itemize}
\item There exists an isometric embedding $\iota: X \to \tilde{X}$.
\item The image $\iota(X) \subset \tilde{X}$ is dense in $X$.
\item $(\tilde{X}, \tilde{d})$ is complete.
\end{itemize}
\textbf{Theorem 1.52} Every metric space has a completion. Furthermore, the completion is unique up to isomorphism.\\
\\
\
\textbf{Proof of Theorem 1.52} This proof is looong. Read the proof in the book, I won't write it here.













\section{Compactness}



















\section{Maxima and minima}
















\section{Exercises}



\textbf{Exercise 1.2} Give an $\epsilon-\delta$ proof that \begin{equation*}
\sum_{n=0} ^\infty x^n = \frac{1}{1 - x},
\end{equation*}
when $|x| < 1$.\\
\\
\
\textbf{Solution} We begin by obtaining a nice expression for the $n$th partial sum \begin{equation*}
s_n (x) = \sum_{n=0} ^k x^k = 1 + x + x^2 + \cdots + x^k.
\end{equation*}
To do so, observe that \begin{align*}
x s_n (x)& = x \sum_{n=0} ^k x^k \\
\ & = x + x^2 + \cdots + x^{k+1} \\
\ & = s_n (x) + x^{k+1} - 1.
\end{align*}
Solving for $s_n(x)$ gives \begin{equation*}
s_n(x) = \frac{1- x^{k+1} }{1 - x}.
\end{equation*}
In other words, \begin{align*}
s_n(x) & = \frac{1- x^{k+1} }{1 - x} \\
\ & = \frac{1}{1 - x} - \frac{x^{k+1}}{1 - x}.
\end{align*}
So it finally suffices to show that $\lim_{n \to \infty} x^n = 0$ for $|x| < 1$. For $\epsilon > 0$, we need to find $N \in \N$ such that $n \geq N$ implies $|x^n| < \epsilon$. To do so, simply set $N > \log_x (\epsilon)$. QED\\
\\
\
\textbf{Exercise 1.3 (Nov 25)} If $x, y, z$ are points in a metric space $(X, d)$, show that \begin{equation*}
d(x,y) \geq |d(x,z) - d(y,z)|
\end{equation*}
\textbf{Solution} The triangle inequality tells us that \begin{equation*}
d(x, z) \geq d(x, y) + d(y, z)
\end{equation*}
and \begin{equation*}
d(y, z) \geq d(x, y) + d(x, z).
\end{equation*}
Isolating $d(x,y)$ in both inequalities gives \begin{align*}
d(x,y) & \leq d(x,z) - d(y,z), \\
d(x,y) & \leq - ((d,z) - d(y,z)).
\end{align*}
Equivalently, \begin{equation*}
d(x,y) \leq |d(x,z) - d(y,z)|.
\end{equation*}
QED\\
\\
\textbf{Exercise 1.4 (Nov 26)} Suppose that $(X, d_X)$ and $(Y, d_Y)$ are metric spaces. Prove that the Cartesian product $Z = X \times Y$ is a metric space with metric $d: Z \times Z \to \R$ defined by \begin{equation*}
d(z_1, z_2) = d_X(x_1, x_2) + d_Y(y_1, y_2)
\end{equation*}
where $z_i = (x_i, y_i)$.\\
\\
\
\textbf{Solution} Without any actual calculation we have \begin{equation*}
d(z_1, z_2) = d(z_2, d_1) \mbox{ (symmetry)}.
\end{equation*}
Since $d_X$ and $d_Y$ only give nonnegative values, so does $d$. \begin{equation*}
d(z_1, z_2) \geq 0.
\end{equation*}
And for the same reason, $d$ vanishes if and only if $d_X$ and $d_Y$ do \begin{equation*}
d(z_1, z_2) = 0 \mbox{ iff } z_1 = z_2.
\end{equation*}
As for the triangle inequality, we already know that \begin{align*}
d_X(x_1, x_3) \leq d_X(x_1, x_2) + d_X(x_2, x_3), \\
d_Y(y_1, y_3) \leq d_Y(y_1, y_2) + d_Y(y_2, y_3).
\end{align*}
Adding these two equations gives \begin{equation*}
d(z_1, z_3) \leq d(z_1, z_2) + d(z_2, z_3).
\end{equation*}
QED\\
\\
\
\textbf{Exercise 1.8 (Nov 27)} Let $(x_n)$ be a bounded sequence of real numbers. \\
\\
\
\textbf{Part (a)} Prove that for every $\epsilon > 0$ and every $N \in \N$ there are $n_1, n_2 \geq N$, such that \begin{equation*}
\lim \sup x_n \leq x_{n_1} + \epsilon, \ x_{n_2} - \epsilon \leq \lim \inf x_n .
\end{equation*}
\textbf{Solution to (a)} Notice that we only need to prove the statement about $\lim \sup$, since the proof for $\lim \inf$ is logically identical. We introduce the nicer notation \begin{equation*}
\mbox{sup} _n = \sup \{ x_k \ | \ k \geq n \}
\end{equation*}
so that \begin{equation*}
\lim \sup x_n = \lim_{n \to \infty} \mbox{sup}_n
\end{equation*}
and similarly for inf$_n$ and $\lim \inf$. Now fix $\epsilon > 0$ and $N \in \N$. We want to prove that there exists $n_1 \geq N$ such that \begin{equation*}
\lim \sup x_n \leq {n_1} + \epsilon.
\end{equation*}
Suppose that were false. Then we would have \begin{equation*}
\lim \sup x_n - \epsilon > x_{n_1}
\end{equation*}
for all $n_1 \geq N$. In other words, all terms past the $N$th term would be bounded above by $\lim \sup x_n - \epsilon$. But then we would have sup$_n$ and we would have \begin{equation*}
\mbox{sup}_{n_1} \leq  \lim \sup x_n - \epsilon
\end{equation*}
for all $n_1 \geq N$. This contraditions the fact that sup$_{n_1} \to \lim \sup x_n$ as $n_1 \to \infty$, which must be true by definition, so our supposition is false. QED\\
\\
\
\textbf{Part (b)} Prove that for every $\epsilon > 0$ there exists $N \in \N$ such that \begin{align*}
x_m \leq \lim \sup x_n + \epsilon, \\
x_m \geq \lim \inf x_n - \epsilon
\end{align*}
for all $m \geq N$.\\
\\
\
\textbf{Solution to (b)} Again by symmetry we only need to prove the first statement. So fix $\epsilon > 0$. From the definition \begin{equation*}
\lim \sup x_n = \lim_{m \to \infty} \mbox{sup}_m = \lim _{m \to \infty} \sup \{ x_k \ | \ k \geq m \}
\end{equation*}
and the definition of a limit, there must exist $N \in \N$ such that \begin{equation*}
m \geq N \implies |\mbox{sup}_m - \lim \sup x_n | < \epsilon.
\end{equation*}
Since $(\mbox{sup}_m)$ is monotic decreasing, the absolute value bars are superfluous: \begin{equation*}
m \geq N \implies \mbox{sup}_m - \lim \sup x_n < \epsilon.
\end{equation*}
And by definition we have \begin{equation*}
x_m \leq \mbox{sup}_m.
\end{equation*}
So we conclude that, for all $m \geq N$, \begin{equation*}
x_m - \lim \sup x_n < \epsilon.
\end{equation*}
QED\\
\\
\
\textbf{Part (c)} Prove that $(x_n)$ converges if and only if \begin{equation*}
\lim \inf x_n = \lim \sup x_n
\end{equation*}
\textbf{Solution to (c)} One direction of this equivalence amounts to the sandwich thereom: Assume $\lim \inf x_n = \lim \sup x_n$ and call the common limit $L \in \R$. Observe the convenient bounding \begin{equation*}
\mbox{inf}_n \leq x_n \leq \mbox{sup}_n \mbox{ for all } n \in N.
\end{equation*}
Then for all $\epsilon > 0$, there exists $N \in \N$ such that \begin{equation*}
n \geq N \implies L - \epsilon < \mbox{inf}_n \leq x_n \leq \mbox{sup}_n \leq L + \epsilon
\end{equation*}
and we have \begin{equation*}
\lim _{n \to \infty} x_n = L.
\end{equation*}
The opposite direction follows immediately from parts (a) and (b). QED\\
\\
\
\textbf{Exercise 1.11 (Nov 28)} If $(x_n), (a_n), (b_n)$ are sequences of real numbers such that \begin{equation*}
\lim_{n \to \infty} x_n = x
\end{equation*}
and such that \begin{equation*}
a_n \leq x_n \leq b_n \ \forall n \in \N
\end{equation*}
prove that \begin{equation*}
\lim \sup a_n \leq x \leq \lim \inf b_n.
\end{equation*}
\textbf{Solution} First, observe that $(x_n)$ converges and is thus bounded, so $(a_n)$ is bounded above and $(b_n)$ is bounded below. This still allows for the possibility that $\lim \sup a_n = - \infty$ or $\lim \inf b_n = \infty$, but in those cases our claimed inequalities are trivially satisfied. So without loss of generality, we assume that $\lim \sup a_n \in \R$ and $\lim \inf b_n \in \R$, or equivalently that $(a_n)$ is also bounded below and $(b_n)$ above.\\
\\
\
From the inequality \begin{equation*}
a_n \leq x_n \ \forall n \in \N
\end{equation*}
we see that \begin{equation*}
\sup \{a_k \ | \ k \geq n \} \leq \sup \{x_k \ | \ k \geq n \}
\end{equation*}
which holds in the limit $n \to \infty$. \begin{equation*}
\lim \sup a_n \leq \lim \sup x_n.
\end{equation*}
But since $(x_n)$ is a convergent series we know its $\lim \sup$ agrees with its limit: \begin{equation*}
\lim \sup x_n = x.
\end{equation*}
So we conclude \begin{equation*}
\lim \sup a_n \leq x.
\end{equation*}
By identical reasoning, we have \begin{equation*}
x \leq \lim \inf b_n.
\end{equation*}
QED\\
\\
\
\textbf{Exercise 1.12 (Nov 28)} Let $(X, d_X)$, $(Y, d_Y)$, and $(Z, d_Z)$ be metric spaces and let $f: X \to Y$ and $g: Y \to Z$ be continuous functions. Show that the composition \begin{equation*}
h = g \circ f: X \to Z
\end{equation*}
is also continuous.\\
\\
\
\textbf{Solution} This problem is messy if we use the $\epsilon-\delta$ criterion for continuity but easy if we use the topological criterion: a function is continuous iff its preimages of open sets are open. So let $U \subset Z$ be open and let $V \subset X$ be its preimage under $h$. We wish to show that $V$ is open. But we have \begin{align*}
V & = h^{-1}(U) \\
\ & = (g \circ f)^{-1}(U) \\
\ & = f^{-1} (g^{-1}(U))
\end{align*}
$g$ is continuous, so $g^{-1}(U) \subset Y$ is open in $Y$. And $f$ is continuous, so $f^{-1} (g^{-1}(U)) \subset X$ is open in $X$. QED


\end{document}
